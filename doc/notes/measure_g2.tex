\documentclass[a4paper,12pt]{article}
\usepackage[utf8]{inputenc}
\usepackage[margin=0.7in]{geometry}
\usepackage{amsmath,amssymb}
\usepackage{caption}
\usepackage{verbatim}
\usepackage{hyperref}
\DeclareMathAlphabet{\mathpzc}{OT1}{pzc}{m}{it}

\newcommand{\aver}[1]{\ensuremath{\langle#1\rangle}}
\renewcommand{\t}{\ensuremath{\tau}}
\newcommand{\w}{\ensuremath{\omega}}
\newcommand{\W}{\ensuremath{\Omega}}
\newcommand{\n}{\ensuremath{\nu}}
\newcommand{\TT}{\ensuremath{\mathcal{T}_\t}}

\begin{document}
\title{Implementation notes: accumulation of two-particle Green's functions in \texttt{CTHYB}}
\author{Igor Krivenko}
\maketitle

\section{General notations}
\begin{itemize}
	\item $\nu_n = \pi(2n+1)/\beta$ --- fermionic Matsubara frequency;
	\item $\w_m = 2\pi m/\beta$ -- bosonic Matsubara frequency;
	\item $P_\ell[x(\t)]$ --- Legendre polynomials of $x$, $x(\t) = 2\t/\beta - 1$;
	\item $\sum_{\nu_n} = \frac{1}{\beta}\sum_{n=-\infty}^{+\infty}$ ---
		  Matsubara frequency summation.
\end{itemize}

\section{Single-particle Green's function}

Definition:
\begin{equation}
	G_{\alpha\beta}(\t_1-\t_2) = -\aver{\TT c_\alpha(\t_1) c^\dag_\beta(\t_2)}.
\end{equation}

Matsubara-frequency representation:
\begin{equation}
	G_{\alpha\beta}(i\nu_n) = \int_0^\beta d\t e^{i\nu_n\t} G_{\alpha\beta}(\t),
	\quad
	G_{\alpha\beta}(\t) = \sum_{\nu_n} e^{-i\nu_n} G_{\alpha\beta}(i\nu_n).
\end{equation}

Legendre representation:
\begin{equation}
	G_{\alpha\beta}(\ell) = \sqrt{2\ell+1}\int_0^\beta d\t P_\ell[x(\t)] G_{\alpha\beta}(\t),
	\quad
	G_{\alpha\beta}(\t) = \sum_{\l\ge 0} \frac{\sqrt{2\l+1}}{\beta} P_\ell[x(\t)] G_{\alpha\beta}(\ell).
\end{equation}

\section{Two-particle Green's function}
\subsection{Definition}

I define a general two-particle Green's function in imaginary-time following the notation adopted in
\cite{Boehnke15} and \cite{RohringerThesis}.
\begin{equation}
	G^{(2)}_{\alpha\beta\gamma\delta}(\t_1,\t_2,\t_3,\t_4) =
	\aver{\mathcal{T}_\t c^\dag_\alpha(\t_1) c_\beta(\t_2) c^\dag_\gamma(\t_3) c_\delta(\t_4)}.
\end{equation}

A configuration of the hybridization expansion algorithm is defined by the
perturbation order $K$, the pairs of imaginary time points where $c$ and $c^\dag$
operators are inserted, and their corresponding indices,
\begin{equation}
\mathcal{C} = (K;\{\tau_1\lambda_1,\tau'_1\lambda'_1;\ldots;\tau_K\lambda_K,\tau'_K\lambda'_K\}).
\end{equation}
The hybridization processes contribute to the Monte-Carlo weight of the configuration
$\mathcal{C}$ as $|\det \mathbf{\hat\Delta}(\mathcal{C})|$ factor, where
$\mathbf{\Delta}(\mathcal{C})_{ij} \equiv \Delta_{\lambda_i,\lambda'_j}(\tau_i-\tau'_j)$.
The accumulation procedures involve the inverted matrix $\mathbf{\hat M} = \mathbf{\hat\Delta}^{-1}$.

$G^{(2)}$ can in principle be accumulated directly in the imaginary time representation,
\begin{multline}\label{accum_tau}
    G^{(2)}_{\alpha\beta\gamma\delta}(\t_1,\t_2,\t_3,\t_4) =
    \frac{1}{\beta}\left\langle
    \sum_{ijkl=1}^K
    (\mathbf{M}_{ij} \mathbf{M}_{kl} - \mathbf{M}_{il} \mathbf{M}_{kj})
    \delta_{\lambda'_i,\alpha} \delta_{\lambda_j,\beta}
    \delta_{\lambda'_k,\gamma} \delta_{\lambda_l,\delta}
    \times\right.\\\left.\times
    \delta^-[\t_1 - \t_2 -(\t'_i-\t_j)]\delta^-[\t_3-\t_4-(\t'_k-\t_l)]
    \delta^+[\t_1-\t_4-(\tau'_i-\tau_l)]
    \vphantom{\sum_{ijkl=1}^K} \right\rangle_\mathrm{MC},
\end{multline}
with $\delta^{\pm}$ being periodized/anti-periodized delta-function,
\begin{equation}
    \delta^{\pm}(\tau) = \sum_{n\in\mathbb{Z}}(\pm1)^n\delta(\tau-n\beta).
\end{equation}


\subsection{Block structure of $G^{(2)}$}

In the previous paragraph I used Greek letters to denote operator indices without paying attention to the block structure of the hybridization function.
It is, however, important to treat a general index $\alpha=(A;a)$ as a combination of a
block index $A$ and an internal index $a$ in order to analyze the block structure of
$G^{(2)}$. The matrix $\mathbf{M}$ is block diagonal with the same blocks as $\hat \Delta(\t)$. Therefore, only the following two block index combinations can receive
non-vanishing contributions during accumulation,
\begin{multline}\label{accum_AABB}
	G^{(2)AABB}_{abcd}(\t_1,\t_2,\t_3,\t_4) =
	\frac{1}{\beta}\left\langle
	\sum_{ij=1}^{K_A}\sum_{kl=1}^{K_B}
	(\mathbf{M}^A_{ij} \mathbf{M}^B_{kl} - \delta_{AB}\mathbf{M}^A_{il} \mathbf{M}^A_{kj})
	\times\right.\\\left.\times
	\delta_{\lambda'_i,a} \delta_{\lambda_j,b}\delta_{\lambda'_k,c} \delta_{\lambda_l,d}
	\delta^-[\t_1 - \t_2 -(\t'_i-\t_j)]\delta^-[\t_3-\t_4-(\t'_k-\t_l)]
	\delta^+[\t_1-\t_4-(\tau'_i-\tau_l)]
	\vphantom{\sum_{ijkl=1}^K} \right\rangle_\mathrm{MC},
\end{multline}
\begin{multline}\label{accum_ABBA}
	G^{(2)ABBA}_{abcd}(\t_1,\t_2,\t_3,\t_4) =
	\frac{1}{\beta}\left\langle
	\sum_{il=1}^{K_A}\sum_{kj=1}^{K_B}
	(\delta_{AB}\mathbf{M}^A_{ij} \mathbf{M}^A_{kl} - \mathbf{M}^A_{il} \mathbf{M}^B_{kj})
	\times\right.\\\left.\times
	\delta_{\lambda'_i,a} \delta_{\lambda_j,b}\delta_{\lambda'_k,c} \delta_{\lambda_l,d}
	\delta^-[\t_1 - \t_2 -(\t'_i-\t_j)]\delta^-[\t_3-\t_4-(\t'_k-\t_l)]
	\delta^+[\t_1-\t_4-(\tau'_i-\tau_l)]
	\vphantom{\sum_{ijkl=1}^K} \right\rangle_\mathrm{MC}.
\end{multline}

The choice between the two notations is dictated by the physical quantity of interest.
For example, in many cases the block indices correspond to spin projections,
$A=\uparrow,\downarrow$. One would prefer the $AABB$ notation to compute a
density-density correlator $\langle n_\sigma(\t)n_{\sigma'}(0)\rangle$ from the results
of $G^{(2)}$ accumulation. On the other hand, the $ABBA$ notation would be more
convenient for a calculation of the transversal magnetic susceptibility $\langle S_+(\t)S_-(0) \rangle$.

In order to minimize memory consumption, the accumulated Green's function will be stored
in a \texttt{block2\_gf} container with indices $A,B$. Interpretation of this
pair of indices
as either $AABB$ or $ABBA$ will be controlled via a switchable option. There will also be
an option to measure a subset of all possible $AB$ pairs.

\subsection{Accumulation at Matsubara frequencies: $ph$- and $pp$-channels}

It is rather impractical to accumulate in the imaginary time domain, because the number of
time slices along each of the three $\t$-directions (one dimension is eliminated by
the time shift symmetry) is strongly limited. Typically one has enough memory to store only
$10^1-10^2$ slices, which would provide a very rough description of the actual function.
The same number of data points in the Matsubara frequency representation would contain
much more relevant information.

It will be possible to switch between the \textit{particle-hole} and \textit{particle-particle}
notation for the Fourier transform.

\begin{equation}
	G^{(2)ph}_{\alpha\beta\gamma\delta}(\w;\nu,\nu') =
	\frac{1}{\beta}\int_0^\beta d\t_1d\t_2d\t_3d\t_4\
	e^{-i\nu\t_1} e^{i(\nu+\w)\t_2} e^{-i(\nu'+\w)\t_3} e^{i\nu'\tau_4}
	G^{(2)}_{\alpha\beta\gamma\delta}(\t_1,\t_2,\t_3,\t_4).
\end{equation}
\begin{equation}
	G^{(2)pp}_{\alpha\beta\gamma\delta}(\w;\nu,\nu') =
	\frac{1}{\beta}\int_0^\beta d\t_1d\t_2d\t_3d\t_4\
	e^{-i\nu\t_1} e^{i(\w-\nu')\t_2} e^{-i(\w-\nu)\t_3} e^{i\nu'\tau_4}
	G^{(2)}_{\alpha\beta\gamma\delta}(\t_1,\t_2,\t_3,\t_4).
\end{equation}

Let us apply these Fourier transformations to the product of $\delta^\pm$-functions in
(\ref{accum_tau}).

\begin{multline}\label{ph_fourier_kernel}
	\frac{1}{\beta}\int_0^\beta d\t_1d\t_2d\t_3d\t_4\
	e^{-i\nu\t_1} e^{i(\nu+\w)\t_2} e^{-i(\nu'+\w)\t_3} e^{i\nu'\tau_4}\times\\\times
    \delta^-[\t_1 - \t_2 -(\t'_i-\t_j)]\delta^-[\t_3-\t_4-(\t'_k-\t_l)]
    \delta^+[\t_1-\t_4-(\tau'_i-\tau_l)] =\\=
    \exp[-i\nu\t_i' +i(\nu+\w)\t_j - i(\nu'+\w)\t'_k + i\nu'\t_l].
\end{multline}
\begin{multline}\label{pp_fourier_kernel}
	\frac{1}{\beta}\int_0^\beta d\t_1d\t_2d\t_3d\t_4\
	e^{-i\nu\t_1} e^{i(\w-\nu')\t_2} e^{-i(\w-\nu)\t_3} e^{i\nu'\tau_4}\times\\\times
	\delta^-[\t_1 - \t_2 -(\t'_i-\t_j)]\delta^-[\t_3-\t_4-(\t'_k-\t_l)]
	\delta^+[\t_1-\t_4-(\tau'_i-\tau_l)] =\\=
	\exp[-i\nu\t_i' +i(\w-\nu')\t_j - i(\w-\nu)\t'_k + i\nu'\t_l].
\end{multline}

This result allows me to write down $G^{(2)}$ accumulation formulas for all combinations
of $AB$-orderings and channels.

\begin{itemize}
	\item \textbf{AABB-ph}
	\begin{multline}
		G^{(2),AABB,ph}_{abcd}(\w;\nu,\nu') =\\= \frac{1}{\beta}\left\langle
			\left(\sum_{ij=1}^{K_A} \mathbf{M}^A_{ij}
				\delta_{\lambda'_i,a} \delta_{\lambda_j,b}
				e^{-i\nu\t_i'+i(\nu+\w)\t_j}
			\right)
			\left(\sum_{kl=1}^{K_B} \mathbf{M}^B_{kl}
				\delta_{\lambda'_k,c} \delta_{\lambda_l,d}
				e^{-i(\nu'+\w)\t_k'+i\nu'\t_l}
			\right) -\right.\\\left.-\delta_{AB}
			\left(\sum_{il=1}^{K_A} \mathbf{M}^A_{il}
				\delta_{\lambda'_i,a} \delta_{\lambda_l,d}
				e^{-i\nu\t_i'+i\nu'\t_l}
			\right)
			\left(\sum_{kj=1}^{K_A} \mathbf{M}^A_{kj}
				\delta_{\lambda'_k,c} \delta_{\lambda_j,b}
				e^{-i(\nu'+\w)\t_k'+i(\nu+\w)\t_j}
			\right)
		\right\rangle_\mathrm{MC}.
	\end{multline}

	\item \textbf{ABBA-ph}
	\begin{multline}
		G^{(2),ABBA,ph}_{abcd}(\w;\nu,\nu') =\\= \frac{1}{\beta}\left\langle
			\delta_{AB}
			\left(\sum_{ij=1}^{K_A} \mathbf{M}^A_{ij}
				\delta_{\lambda'_i,a} \delta_{\lambda_j,b}
				e^{-i\nu\t_i'+i(\nu+\w)\t_j}
			\right)
			\left(\sum_{kl=1}^{K_A} \mathbf{M}^A_{kl}
				\delta_{\lambda'_k,c} \delta_{\lambda_l,d}
				e^{-i(\nu'+\w)\t_k'+i\nu'\t_l}
			\right) -\right.\\\left.-
			\left(\sum_{il=1}^{K_A} \mathbf{M}^A_{il}
				\delta_{\lambda'_i,a} \delta_{\lambda_l,d}
				e^{-i\nu\t_i'+i\nu'\t_l}
			\right)
			\left(\sum_{kj=1}^{K_B} \mathbf{M}^B_{kj}
				\delta_{\lambda'_k,c} \delta_{\lambda_j,b}
				e^{-i(\nu'+\w)\t_k'+i(\nu+\w)\t_j}
			\right)
			\right\rangle_\mathrm{MC}.
	\end{multline}

	\item \textbf{AABB-pp}
	\begin{multline}
	G^{(2),AABB,pp}_{abcd}(\w;\nu,\nu') =\\= \frac{1}{\beta}\left\langle
		\left(\sum_{ij=1}^{K_A} \mathbf{M}^A_{ij}
			\delta_{\lambda'_i,a} \delta_{\lambda_j,b}
			e^{-i\nu\t_i'+i(\w-\nu')\t_j}
		\right)
		\left(\sum_{kl=1}^{K_B} \mathbf{M}^B_{kl}
			\delta_{\lambda'_k,c} \delta_{\lambda_l,d}
			e^{-i(\w-\nu)\t_k'+i\nu'\t_l}
		\right) -\right.\\\left.-\delta_{AB}
		\left(\sum_{il=1}^{K_A} \mathbf{M}^A_{il}
			\delta_{\lambda'_i,a} \delta_{\lambda_l,d}
			e^{-i\nu\t_i'+i\nu'\t_l}
		\right)
		\left(\sum_{kj=1}^{K_A} \mathbf{M}^A_{kj}
			\delta_{\lambda'_k,c} \delta_{\lambda_j,b}
			e^{-i(\w-\nu)\t_k'+i(\w-\nu')\t_j}
		\right)
		\right\rangle_\mathrm{MC}.
	\end{multline}

	\item \textbf{ABBA-pp}
	\begin{multline}
	G^{(2),ABBA,pp}_{abcd}(\w;\nu,\nu') =\\= \frac{1}{\beta}\left\langle
		\delta_{AB}
		\left(\sum_{ij=1}^{K_A} \mathbf{M}^A_{ij}
			\delta_{\lambda'_i,a} \delta_{\lambda_j,b}
			e^{-i\nu\t_i'+i(\w-\nu')\t_j}
		\right)
		\left(\sum_{kl=1}^{K_A} \mathbf{M}^A_{kl}
			\delta_{\lambda'_k,c} \delta_{\lambda_l,d}
			e^{-i(\w-\nu)\t_k'+i\nu'\t_l}
		\right) -\right.\\\left.-
		\left(\sum_{il=1}^{K_A} \mathbf{M}^A_{il}
			\delta_{\lambda'_i,a} \delta_{\lambda_l,d}
			e^{-i\nu\t_i'+i\nu'\t_l}
		\right)
		\left(\sum_{kj=1}^{K_B} \mathbf{M}^B_{kj}
			\delta_{\lambda'_k,c} \delta_{\lambda_j,b}
			e^{-i(\w-\nu)\t_k'+i(\w-\nu')\t_j}
		\right)
		\right\rangle_\mathrm{MC}.
	\end{multline}
\end{itemize}

It is worth noting that none of the formulas above contain a four-fold sum
over elements of the $\mathbf{M}$ matrices. Factorization of such sums
into products of double sums allows to significantly speed up the accumulation
process.

\subsection{Accumulation in the mixed Matsubara/Legendre basis}

Lewin Boehnke proves in his thesis \cite{Boehnke15} that a modified (shifted)
transformation to the Legendre basis should be used in order to minimize the
number of required coefficients. Matrix inversion of quantities in this basis
is also more stable.

\begin{equation}\label{legendre_transform}
	G^{(2)}_{\alpha\beta\gamma\delta}(\w_m;\ell,\ell') \equiv
	\sum_{n,n'\in\mathbb{Z}}
	\bar T_{2n+m+1,\ell}
	G^{(2)}_{\alpha\beta\gamma\delta}(\w_m;\nu_n,\nu_{n'})
	\bar T^*_{2n'+m+1,\ell'},
\end{equation}
\begin{equation}
	\bar T_{o,\ell} \equiv \frac{\sqrt{2\ell+1}}{\beta}
	\int_0^\beta d\t e^{io\pi\frac{\t}{\beta}} P_\ell[x(\t)] =
	\sqrt{2\ell+1}i^o i^\ell j_\ell\left(\frac{o\pi}{2}\right).
\end{equation}

Applying transformation (\ref{legendre_transform}) to the right hand parts of
equations (\ref{ph_fourier_kernel}) and (\ref{pp_fourier_kernel}) we obtain

\begin{multline}
	\sum_{n,n'\in\mathbb{Z}}
	\bar T_{2n+m+1,\ell}
	e^{-i\nu_n(\t'_i-\t_j) -i\w_m(\t'_k-\t_j) -i\nu_{n'}(\t'_k-\t_l)}
	\bar T^*_{2n'+m+1,\ell'} =\\=
	(-1)^{\ell'+1}\sqrt{2\ell+1}\sqrt{2\ell'+1}
	\tilde P_\ell[x(\t'_i-\t_j)] \tilde P_{\ell'}[x(\t'_k-\t_l)]
	e^{i\frac{\w_m}{2}(\t'_i-\t_j+\t'_k-\t_l)} e^{-i\w_m(\t'_k-\t_j)},
\end{multline}
\begin{multline}
	\sum_{n,n'\in\mathbb{Z}}
	\bar T_{2n+m+1,\ell}
	e^{-i\nu_n(\t'_i-\t'_k) -i\w_m(\t'_k-\t_j) -i\nu_{n'}(\t_j-\t_l)}
	\bar T^*_{2n'+m+1,\ell'} =\\=
	(-1)^{\ell'+1}\sqrt{2\ell+1}\sqrt{2\ell'+1}
	\tilde P_\ell[x(\t'_i-\t'_k)] \tilde P_{\ell'}[x(\t_j-\t_l)]
	e^{i\frac{\w_m}{2}(\t'_i-\t'_k+\t_j-\t_l)} e^{-i\w_m(\t'_k-\t_j)},
\end{multline}
\begin{equation}
	\tilde P_\ell[x(\delta\t)] \equiv \left\{
	\begin{array}{ll}
	P_\ell[x(\delta\t)], &\delta\t>0,\\
	-P_\ell[x(\delta\t+\beta)], &\delta\t<0.
	\end{array}
	\right.
\end{equation}

Accumulation formulas in the mixed Matsubara/Legendre basis are given below.
\begin{itemize}
	\item \textbf{AABB-ph}
	\begin{multline}
		G^{(2)AABB}_{abcd}(\w;\ell,\ell') =
		\frac{(-1)^{\ell'+1}\sqrt{2\ell+1}\sqrt{2\ell'+1}}{\beta}\left\langle
		\sum_{ij=1}^{K_A}\sum_{kl=1}^{K_B}
		(\mathbf{M}^A_{ij} \mathbf{M}^B_{kl} - \delta_{AB}\mathbf{M}^A_{il} \mathbf{M}^A_{kj})
		\times\right.\\\left.\times
		\delta_{\lambda'_i,a} \delta_{\lambda_j,b}\delta_{\lambda'_k,c} \delta_{\lambda_l,d}
		\tilde P_\ell[x(\t'_i-\t_j)] \tilde P_{\ell'}[x(\t'_k-\t_l)]
		e^{i\frac{\w}{2}(\t'_i-\t_j+\t'_k-\t_l)} e^{-i\w(\t'_k-\t_j)}
		\vphantom{\sum_{ijkl=1}^K} \right\rangle_\mathrm{MC}.
	\end{multline}
	\item \textbf{ABBA-ph}
	\begin{multline}
		G^{(2)ABBA}_{abcd}(\w;\ell,\ell') =
		\frac{(-1)^{\ell'+1}\sqrt{2\ell+1}\sqrt{2\ell'+1}}{\beta}\left\langle
		\sum_{il=1}^{K_A}\sum_{kj=1}^{K_B}
		(\delta_{AB}\mathbf{M}^A_{ij} \mathbf{M}^A_{kl} - \mathbf{M}^A_{il} \mathbf{M}^B_{kj})
		\times\right.\\\left.\times
		\delta_{\lambda'_i,a} \delta_{\lambda_j,b}\delta_{\lambda'_k,c} \delta_{\lambda_l,d}
		\tilde P_\ell[x(\t'_i-\t_j)] \tilde P_{\ell'}[x(\t'_k-\t_l)]
		e^{i\frac{\w}{2}(\t'_i-\t_j+\t'_k-\t_l)} e^{-i\w(\t'_k-\t_j)}
		\vphantom{\sum_{ijkl=1}^K} \right\rangle_\mathrm{MC}.
	\end{multline}
	\item \textbf{AABB-pp}
	\begin{multline}
		G^{(2)AABB}_{abcd}(\w;\ell,\ell') =
		\frac{(-1)^{\ell'+1}\sqrt{2\ell+1}\sqrt{2\ell'+1}}{\beta}\left\langle
		\sum_{ij=1}^{K_A}\sum_{kl=1}^{K_B}
		(\mathbf{M}^A_{ij} \mathbf{M}^B_{kl} - \delta_{AB}\mathbf{M}^A_{il} \mathbf{M}^A_{kj})
		\times\right.\\\left.\times
		\delta_{\lambda'_i,a} \delta_{\lambda_j,b}\delta_{\lambda'_k,c} \delta_{\lambda_l,d}
		\tilde P_\ell[x(\t'_i-\t'_k)] \tilde P_{\ell'}[x(\t_j-\t_l)]
		e^{i\frac{\w}{2}(\t'_i-\t'_k+\t_j-\t_l)} e^{-i\w(\t'_k-\t_j)}
		\vphantom{\sum_{ijkl=1}^K} \right\rangle_\mathrm{MC}.
	\end{multline}
	\item \textbf{ABBA-pp}
	\begin{multline}
		G^{(2)ABBA}_{abcd}(\w;\ell,\ell') =
		\frac{(-1)^{\ell'+1}\sqrt{2\ell+1}\sqrt{2\ell'+1}}{\beta}\left\langle
		\sum_{il=1}^{K_A}\sum_{kj=1}^{K_B}
		(\delta_{AB}\mathbf{M}^A_{ij} \mathbf{M}^A_{kl} - \mathbf{M}^A_{il} \mathbf{M}^B_{kj})
		\times\right.\\\left.\times
		\delta_{\lambda'_i,a} \delta_{\lambda_j,b}\delta_{\lambda'_k,c} \delta_{\lambda_l,d}
		\tilde P_\ell[x(\t'_i-\t'_k)] \tilde P_{\ell'}[x(\t_j-\t_l)]
		e^{i\frac{\w}{2}(\t'_i-\t'_k+\t_j-\t_l)} e^{-i\w(\t'_k-\t_j)}
		\vphantom{\sum_{ijkl=1}^K} \right\rangle_\mathrm{MC}.
	\end{multline}
\end{itemize}

\bibliographystyle{plain}

\begin{thebibliography}{9}

\bibitem{Boehnke15}
    Lewin Volker Boehnke,
    \emph{Susceptibilities in materials with multiple strongly correlated orbitals}
    PhD thesis, Universit\"at Hamburg, 2015,
    \url{http://ediss.sub.uni-hamburg.de/volltexte/2015/7325/pdf/Dissertation.pdf}

\bibitem{RohringerThesis}
    Georg Rohringer,
    \emph{New routes towards a theoretical treatment of nonlocal electronic correlations}
    PhD thesis, Technische Universit\"at Wien, 2013,
    \url{https://www.ifp.tuwien.ac.at/fileadmin/Arbeitsgruppen/quantum_manybody_physics/docs/thesis/RohringerPhD.pdf},

\end{thebibliography}

\end{document}
